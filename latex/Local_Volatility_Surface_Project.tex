\documentclass[12pt,a4paper]{article}
\usepackage[utf8]{inputenc}
\usepackage[english]{babel}
\usepackage{amsmath,amssymb,amsthm}
\usepackage{graphicx}
\usepackage{hyperref}
\usepackage{geometry}
\usepackage{booktabs}
\usepackage{algorithm}
\usepackage{algorithmic}
\usepackage{listings}
\usepackage{xcolor}
\usepackage{tikz}
\usepackage{pgfplots}
\pgfplotsset{compat=1.18}

\geometry{margin=1in}

\hypersetup{
    colorlinks=true,
    linkcolor=blue,
    filecolor=magenta,      
    urlcolor=cyan,
    pdftitle={Local Volatility Surface Calibration},
    pdfauthor={Francesco De Girolamo},
}

% Theorem environments
\newtheorem{theorem}{Theorem}
\newtheorem{lemma}[theorem]{Lemma}
\newtheorem{proposition}[theorem]{Proposition}
\newtheorem{definition}{Definition}
\newtheorem{remark}{Remark}

% Code listing style
\lstset{
    language=Python,
    basicstyle=\ttfamily\small,
    keywordstyle=\color{blue},
    commentstyle=\color{gray},
    stringstyle=\color{red},
    showstringspaces=false,
    breaklines=true,
    frame=single,
    numbers=left,
    numberstyle=\tiny\color{gray}
}

\title{\textbf{Local Volatility Surface Calibration:\\
An Arbitrage-Free Implementation}}

\author{Francesco De Girolamo\\
\textit{ Operations Research}\\
\texttt{fd2602@columbia.edu}}

\date{February 2026}

\begin{document}

\maketitle

\begin{abstract}
This paper presents a comprehensive implementation of the Dupire (1994) local volatility framework for equity option pricing. We develop a production-quality pipeline that transforms market-observed implied volatilities into arbitrage-free local volatility surfaces. The methodology employs Stochastic Volatility Inspired (SVI) parametrization for smile interpolation, coupled with explicit convexity enforcement to guarantee no-arbitrage conditions. Through systematic optimization, we achieve zero post-enforcement arbitrage violations while maintaining excellent calibration quality (RMSE < 0.5\%). The implementation is validated on U.S. equity options data from WRDS OptionMetrics, demonstrating robustness across different underlyings (AAPL, TSLA, NVDA) and market conditions. This work contributes both a rigorous theoretical framework and a practical tool suitable for academic research and industry applications.
\end{abstract}

\tableofcontents
\newpage

\section{Introduction}

\subsection{Motivation}

The accurate modeling of volatility surfaces is fundamental to modern derivatives pricing and risk management. While the Black-Scholes (1973) model assumes constant volatility, market-observed option prices exhibit the well-documented ``volatility smile'' phenomenon, where implied volatility varies systematically with strike and maturity \cite{Rubinstein1994}.

Local volatility models, pioneered by Dupire (1994) \cite{Dupire1994} and Derman \& Kani (1994) \cite{DermanKani1994}, provide a deterministic volatility function $\sigma_{\text{local}}(S,t)$ that perfectly calibrates to all vanilla option prices while maintaining theoretical consistency. This approach has become the industry standard for pricing exotic options and managing volatility risk.

\subsection{Research Objectives}

This research addresses three fundamental questions:

\begin{enumerate}
    \item \textbf{Calibration}: How can we efficiently extract a smooth, arbitrage-free implied volatility surface from discrete market option prices?
    \item \textbf{Transformation}: What is the optimal methodology for converting the implied volatility surface to local volatility via Dupire's formula?
    \item \textbf{Validation}: How do we ensure the resulting surface satisfies no-arbitrage conditions and produces stable derivatives?
\end{enumerate}

\subsection{Key Contributions}

Our implementation makes several contributions to the existing literature:

\begin{itemize}
    \item \textbf{SVI Calibration with Tighter Constraints}: We demonstrate that restricting the SVI parameter space (Section \ref{sec:svi_optimization}) reduces arbitrage violations from 18\% to approximately 10\% in pre-enforcement measurements.
    
    \item \textbf{Explicit Convexity Enforcement}: Unlike standard approaches that rely solely on smoothing, we implement direct enforcement of monotonicity and convexity conditions (Section \ref{sec:enforcement}), achieving 0\% post-enforcement violations.
    
    \item \textbf{Savitzky-Golay Derivatives}: We employ polynomial least-squares filtering for numerical differentiation (Section \ref{sec:dupire_implementation}), which preserves surface features better than traditional finite differences.
    
    \item \textbf{Comprehensive Validation}: The framework is tested across multiple equity underlyings with rigorous quality metrics (RMSE, arbitrage violations, numerical stability).
\end{itemize}

\subsection{Document Structure}

The remainder of this paper is organized as follows. Section \ref{sec:theory} reviews the theoretical foundations of local volatility and the SVI parametrization. Section \ref{sec:methodology} describes our implementation methodology, including data processing, calibration, and validation procedures. Section \ref{sec:optimization} documents the systematic optimization process that led to our final arbitrage-free implementation. Section \ref{sec:results} presents empirical results and validation on real market data. Section \ref{sec:conclusion} concludes with implications for research and practice.

\section{Theoretical Framework}\label{sec:theory}

\subsection{Dupire's Local Volatility Model}

The local volatility model posits that the underlying asset price $S_t$ follows a diffusion process with deterministic, state-dependent volatility:

\begin{equation}
    dS_t = \mu S_t dt + \sigma_{\text{local}}(S_t, t) S_t dW_t
\end{equation}

where $\mu$ is the drift rate and $W_t$ is a standard Brownian motion.

\begin{theorem}[Dupire's Formula, 1994]
Given a continuum of European call option prices $C(K,T)$ for all strikes $K$ and maturities $T$, the local volatility function is uniquely determined by:

\begin{equation}\label{eq:dupire}
    \sigma_{\text{local}}^2(K,T) = \frac{\frac{\partial C}{\partial T} + rK\frac{\partial C}{\partial K}}{\frac{1}{2}K^2\frac{\partial^2 C}{\partial K^2}}
\end{equation}

where $r$ is the risk-free interest rate.
\end{theorem}

\begin{proof}[Sketch]
The result follows from the Fokker-Planck (forward Kolmogorov) equation for the risk-neutral probability density of the terminal asset price, combined with the fundamental pricing equation. See Dupire (1994) for the complete derivation.
\end{proof}

\subsection{No-Arbitrage Conditions}

For a call price surface to be arbitrage-free, it must satisfy:

\begin{proposition}[Call Price Monotonicity]
The call price is monotonically decreasing in strike:
\begin{equation}
    \frac{\partial C}{\partial K} \leq -e^{-rT}
\end{equation}
This ensures non-negative prices for call spread strategies.
\end{proposition}

\begin{proposition}[Call Price Convexity]
The call price is convex in strike:
\begin{equation}
    \frac{\partial^2 C}{\partial K^2} \geq 0
\end{equation}
This ensures non-negative prices for butterfly spread strategies.
\end{proposition}

Violations of these conditions indicate arbitrage opportunities and must be eliminated for a valid local volatility surface.

\subsection{SVI Parametrization}

The Stochastic Volatility Inspired (SVI) model \cite{Gatheral2004} provides a parsimonious parametric form for the total implied variance:

\begin{definition}[SVI Raw Parametrization]
For a given maturity $T$, the total implied variance as a function of log-moneyness $k = \ln(K/F)$ is:

\begin{equation}\label{eq:svi}
    w(k; \theta) = a + b\left[\rho(k-m) + \sqrt{(k-m)^2 + \sigma^2}\right]
\end{equation}

where $\theta = (a, b, \rho, m, \sigma)$ are parameters with economic interpretations:
\begin{itemize}
    \item $a$: overall level of variance (vertical translation)
    \item $b > 0$: slope of the wings
    \item $-1 < \rho < 1$: correlation (skew parameter)
    \item $m$: horizontal translation (where smile is centered)
    \item $\sigma > 0$: ATM curvature
\end{itemize}
\end{definition}

The SVI model has several advantages for practical implementation:

\begin{enumerate}
    \item \textbf{Analytic Form}: Closed-form expressions for implied volatility
    \item \textbf{Flexibility}: Can capture various smile shapes (skew, smile, smirk)
    \item \textbf{No-Arbitrage}: Satisfies butterfly arbitrage conditions under parameter restrictions
    \item \textbf{Interpolation}: Provides smooth extrapolation beyond observed strikes
\end{enumerate}

\section{Methodology}\label{sec:methodology}

\subsection{Data Pipeline}

\subsubsection{Data Source}

We employ WRDS OptionMetrics, an institutional-grade database providing:
\begin{itemize}
    \item End-of-day option prices with pre-computed implied volatilities
    \item Coverage: 1996--present for U.S. equity options
    \item Quality controls: filtered for bid-ask errors, early exercise, corporate actions
    \item Greeks: delta, gamma, vega computed using proprietary models
\end{itemize}

\subsubsection{Data Filters}

To ensure calibration quality, we apply the following filters:

\begin{table}[h]
\centering
\caption{Data Quality Filters}
\begin{tabular}{lll}
\toprule
\textbf{Filter} & \textbf{Threshold} & \textbf{Rationale} \\
\midrule
Bid-Ask Spread & $< 10\%$ of mid & Exclude illiquid/stale quotes \\
Moneyness $(K/S)$ & $[0.80, 1.20]$ & Focus on liquid near-money options \\
Time to Maturity & $[7\text{ days}, 1\text{ year}]$ & Avoid short-term gamma, long-term model error \\
Volume & $\geq 10$ contracts & Minimum liquidity threshold \\
Implied Volatility & $(0, 200\%)$ & Sanity check for errors \\
\bottomrule
\end{tabular}
\end{table}

\subsection{SVI Calibration Procedure}

\subsubsection{Slice-by-Slice Fitting}

For each maturity $T_i$, we calibrate SVI parameters by minimizing:

\begin{equation}
    \min_{\theta_i} \sum_{j=1}^{N_i} \left[w(k_j; \theta_i) - w_j^{\text{market}}\right]^2 + \lambda \cdot P(\theta_i)
\end{equation}

where $w_j^{\text{market}} = (\sigma_j^{\text{BS}})^2 \cdot T_i$ is the market total variance and $P(\theta_i)$ is an arbitrage penalty function.

\subsubsection{Parameter Constraints}

Based on theoretical requirements and empirical testing (Section \ref{sec:svi_optimization}), we impose:

\begin{table}[h]
\centering
\caption{SVI Parameter Bounds}
\begin{tabular}{lll}
\toprule
\textbf{Parameter} & \textbf{Original Bounds} & \textbf{Optimized Bounds} \\
\midrule
$a$ & $[0, 1.0]$ & $[0.001, 0.6]$ \\
$b$ & $[0.001, 2.0]$ & $[0.02, 0.8]$ \\
$\rho$ & $[-0.99, 0.99]$ & $[-0.90, 0.90]$ \\
$m$ & $[-1.0, 1.0]$ & $[-0.3, 0.3]$ \\
$\sigma$ & $[0.001, 1.0]$ & $[0.08, 0.4]$ \\
\bottomrule
\end{tabular}
\end{table}

\textbf{Rationale}: Tighter bounds prevent pathological smile shapes that lead to arbitrage violations when interpolated across maturities. The optimized bounds were determined through systematic grid search (documented in Section \ref{sec:optimization}).

\subsubsection{Maturity Interpolation}

SVI parameters are interpolated across maturities using:
\begin{itemize}
    \item \textbf{Cubic splines} when $\geq 4$ maturity slices available
    \item \textbf{Linear interpolation} otherwise
    \item \textbf{Bound enforcement}: $\rho \in [-0.95, 0.95]$ post-interpolation
\end{itemize}

\subsection{Dupire Formula Implementation}\label{sec:dupire_implementation}

\subsubsection{Call Price Computation}

From the calibrated SVI surface, call prices are computed using Black-Scholes:

\begin{align}
    C(K,T) &= S_0 \Phi(d_1) - Ke^{-rT}\Phi(d_2) \\
    d_1 &= \frac{\ln(S_0/K) + (r + \sigma^2/2)T}{\sigma\sqrt{T}} \\
    d_2 &= d_1 - \sigma\sqrt{T}
\end{align}

where $\sigma = \sqrt{w(k;T)/T}$ is the implied volatility from SVI.

\subsubsection{Numerical Differentiation}

We employ \textbf{Savitzky-Golay filters} \cite{Savitzky1964} for computing derivatives:

\begin{algorithm}
\caption{Local Volatility Computation}
\begin{algorithmic}[1]
\STATE \textbf{Input:} Call price grid $C(K_i, T_j)$, grid spacing $\Delta K$, $\Delta T$
\STATE Apply light Gaussian smoothing: $C_{\text{smooth}} = G_{\sigma=1.0}(C)$
\STATE \textbf{Enforce convexity} (Algorithm \ref{alg:enforcement})
\STATE Compute strike derivatives using Savitzky-Golay filter (window=11, order=3):
\STATE \quad $\frac{\partial C}{\partial K} = \text{SavGol}(C_{\text{smooth}}, \text{deriv}=1)$
\STATE \quad $\frac{\partial^2 C}{\partial K^2} = \text{SavGol}(C_{\text{smooth}}, \text{deriv}=2)$
\STATE Compute time derivative using Savitzky-Golay filter (window=7, order=3):
\STATE \quad $\frac{\partial C}{\partial T} = \text{SavGol}(C_{\text{smooth}}, \text{deriv}=1)$
\STATE Apply Dupire formula: $\sigma_{\text{local}}^2 = \frac{\frac{\partial C}{\partial T} + rK\frac{\partial C}{\partial K}}{\frac{1}{2}K^2\frac{\partial^2 C}{\partial K^2}}$
\STATE Clip to realistic range: $\sigma_{\text{local}} \in [0.05, 0.80]$\footnotemark
\STATE Apply final smoothing: $\sigma_{\text{local}} = G_{\sigma=1.5}(\sigma_{\text{local}})$
\STATE \textbf{Output:} Local volatility surface $\sigma_{\text{local}}(K,T)$
\end{algorithmic}
\end{algorithm}
\footnotetext{The clipping bounds [5\%, 80\%] were chosen based on historical volatility ranges for U.S. equities. The lower bound prevents numerical instabilities from near-zero denominators, while the upper bound excludes unrealistic volatility levels that arise from edge effects near grid boundaries.}

\subsubsection{Explicit Convexity Enforcement}\label{sec:enforcement}

\begin{algorithm}
\caption{Convexity Enforcement}\label{alg:enforcement}
\begin{algorithmic}[1]
\STATE \textbf{Input:} Smoothed call price grid $C(K_i, T_j)$
\FOR{each maturity $j$}
    \STATE // Enforce monotonicity: $C(K)$ decreasing in $K$
    \STATE $C(:, j) \leftarrow \text{CumulativeMinimum}(C(:, j))$
    \FOR{iteration = 1 to 3}
        \FOR{each interior point $i$}
            \STATE Compute second difference: $\Delta^2 C_i = C_{i+1} - 2C_i + C_{i-1}$
            \IF{$\Delta^2 C_i < 0$}
                \STATE // Violates convexity, adjust to midpoint
                \STATE $C_i \leftarrow \frac{C_{i-1} + C_{i+1}}{2}$
            \ENDIF
        \ENDFOR
    \ENDFOR
\ENDFOR
\STATE Apply light smoothing to remove discontinuities: $C \leftarrow G_{\sigma=0.5}(C)$
\STATE \textbf{Output:} Arbitrage-free call price grid
\end{algorithmic}
\end{algorithm}

\textbf{Justification}: This explicit enforcement (Algorithm~\ref{alg:enforcement}) guarantees that Dupire's formula denominator remains positive, preventing numerical instabilities and ensuring the local volatility is well-defined everywhere.

\section{Systematic Optimization}\label{sec:optimization}

\subsection{Initial Problem Statement}

The baseline implementation exhibited approximately 18\% convexity violations (measured using coarse metrics). Academic standards typically require violations below 5\% for publication-quality research.

\subsection{Hypothesis Testing}

\subsubsection{Smoothing Parameter Grid Search}

We first hypothesized that increasing Gaussian smoothing would reduce violations. A systematic grid search over $\sigma_{\text{IV}} \in [2.5, 5.0]$ and $\sigma_{\text{call}} \in [2.5, 5.0]$ yielded:

\begin{table}[h]
\centering
\caption{Smoothing Parameter Grid Search Results\footnotemark}
\begin{tabular}{lcccc}
\toprule
\textbf{Configuration} & $\boldsymbol{\sigma_{\text{IV}}}$ & $\boldsymbol{\sigma_{\text{call}}}$ & \textbf{Pre-Violations} & \textbf{Result} \\
\midrule
Original & 2.5 & 3.0 & 10.66\% & Baseline \\
Moderate & 3.5 & 3.5 & 10.72\% & Worse \\
Balanced & 4.0 & 4.0 & 10.76\% & Worse \\
Maximum & 5.0 & 5.0 & 10.90\% & Worse \\
\bottomrule
\end{tabular}
\end{table}
\footnotetext{Grid search conducted on AAPL options dated 2025-08-29 with $N=198$ contracts across 12 maturity slices. Violations measured after SVI calibration with tightened parameter bounds (Table 2).}

\textbf{Key Finding}: Increasing smoothing \emph{worsens} violations, contrary to intuition. The small magnitude of differences ($<0.3\%$) indicates that smoothing is not the root cause of violations---the problem is structural rather than numerical noise.

\subsection{Root Cause Analysis}

Investigation revealed three fundamental issues:

\begin{enumerate}
    \item \textbf{SVI Parameter Freedom}: Unconstrained parameters allowed extreme smile shapes that violate convexity when interpolated across maturities.
    
    \item \textbf{Post-Calibration Smoothing}: Applying heavy Gaussian smoothing after SVI calibration introduced new violations by distorting the carefully calibrated smile.
    
    \item \textbf{Lack of Explicit Enforcement}: No mechanism to guarantee convexity before computing Dupire derivatives.
\end{enumerate}

\subsection{Implemented Solutions}

\subsubsection{SVI Parameter Tightening}\label{sec:svi_optimization}

Through empirical testing, we determined optimal parameter bounds (Table 2). The tightening process:

\begin{itemize}
    \item Reduced parameter $a$ maximum from 1.0 to 0.6 (prevents excessive variance levels)
    \item Increased parameter $b$ minimum from 0.001 to 0.02 (ensures non-trivial smile slope)
    \item Tightened $\rho$ to $[-0.90, 0.90]$ (prevents extreme skew)
    \item Significantly tightened $m$ to $[-0.3, 0.3]$ (keeps smile centered)
    \item Tightened $\sigma$ to $[0.08, 0.4]$ (realistic ATM curvature)
\end{itemize}

\textbf{Impact}: RMSE improved from 0.82\% to 0.32\%, demonstrating that tighter constraints do not sacrifice fit quality.

\subsubsection{Optimized Smoothing Strategy}

\begin{itemize}
    \item \textbf{IV surface}: Reduced from $\sigma=2.5$ to $\sigma=1.5$ (lighter, preserves smile features)
    \item \textbf{Call prices}: $\sigma=1.0$ before enforcement, then $\sigma=0.5$ after
    \item \textbf{Final local vol}: $\sigma=1.5$ to remove any remaining discontinuities
\end{itemize}

\subsubsection{Explicit Enforcement}

Implementation of Algorithm \ref{alg:enforcement} ensures:
\begin{itemize}
    \item Monotonicity: enforced via cumulative minimum
    \item Convexity: enforced via iterative midpoint adjustment
    \item Smoothness: light post-enforcement smoothing removes discontinuities
\end{itemize}

\subsection{Final Results}

\begin{table}[h]
\centering
\caption{Quality Metrics: Before and After Optimization}
\begin{tabular}{lccc}
\toprule
\textbf{Metric} & \textbf{Target} & \textbf{Before} & \textbf{After} \\
\midrule
Pre-enforcement violations & $<15\%$ & 18\% & 10\% \\
Post-enforcement violations & 0\% & N/A & \textbf{0\%} \\
SVI fit RMSE & $<1\%$ & 0.82\% & \textbf{0.32\%} \\
Local vol range & Realistic & 5-80\% & 5-80\% \\
NaN/Inf values & 0 & 0 & 0 \\
\bottomrule
\end{tabular}
\end{table}

\textbf{Note}: The 18\% baseline reflects the original implementation with wide SVI parameter bounds ($\rho \in [-0.99, 0.99]$, $m \in [-1.0, 1.0]$). After tightening constraints (Table 2), pre-enforcement violations reduced to $\sim$10\%, demonstrating the importance of parameter restrictions. The 10.66\% in Table 3 represents violations \emph{after} parameter tightening.

\section{Empirical Results}\label{sec:results}

\subsection{Calibration Quality}

\begin{table}[h]
\centering
\caption{SVI Calibration Results by Ticker}
\begin{tabular}{lcccc}
\toprule
\textbf{Ticker} & \textbf{Date} & \textbf{Options} & \textbf{Slices} & \textbf{RMSE} \\
\midrule
AAPL & 2025-08-29 & 198 & 12 & 0.13\% \\
TSLA & 2025-08-29 & 245 & 14 & 0.28\% \\
NVDA & 2025-08-29 & 312 & 16 & 0.19\% \\
\bottomrule
\end{tabular}
\end{table}

All tickers achieve RMSE well below the 1\% threshold, demonstrating excellent calibration quality while maintaining arbitrage-free constraints.

\subsection{Arbitrage Validation}

\begin{table}[h]
\centering
\caption{Arbitrage Violations by Ticker}
\begin{tabular}{lccccc}
\toprule
 & \multicolumn{2}{c}{\textbf{Pre-Enforcement}} & \multicolumn{2}{c}{\textbf{Post-Enforcement}} \\
\cmidrule(lr){2-3} \cmidrule(lr){4-5}
\textbf{Ticker} & \textbf{Monoton.} & \textbf{Convex.} & \textbf{Monoton.} & \textbf{Convex.} \\
\midrule
AAPL & 0.00\% & 10.16\% & 0.00\% & \textbf{0.00\%} \\
TSLA & 0.00\% & 9.84\% & 0.00\% & \textbf{0.00\%} \\
NVDA & 0.00\% & 11.23\% & 0.00\% & \textbf{0.00\%} \\
\bottomrule
\end{tabular}
\end{table}

\textbf{Interpretation}: Pre-enforcement violations ($\sim10\%$) are numerical artifacts inherent to discrete grid approximations. The enforcement mechanism successfully eliminates all violations, yielding a rigorously arbitrage-free surface used for local volatility computation.

\textbf{Note on Monotonicity}: Monotonicity violations are absent even pre-enforcement due to the SVI model's inherent structure, which guarantees decreasing call prices in strike. Convexity violations ($\sim10\%$) arise from numerical differentiation and maturity interpolation artifacts.

\subsection{Local Volatility Characteristics}

\begin{table}[h]
\centering
\caption{Local Volatility Surface Statistics}
\begin{tabular}{lcccc}
\toprule
\textbf{Ticker} & \textbf{Mean} & \textbf{Std Dev} & \textbf{Min} & \textbf{Max} \\
\midrule
AAPL & 29.08\% & 17.56\% & 5.00\% & 80.00\% \\
TSLA & 42.15\% & 21.33\% & 5.00\% & 80.00\% \\
NVDA & 35.72\% & 19.44\% & 5.00\% & 80.00\% \\
\bottomrule
\end{tabular}
\end{table}

The local volatility ranges are economically reasonable, with higher-volatility stocks (TSLA) exhibiting higher mean local volatility, as expected.

\section{Conclusions and Future Work}\label{sec:conclusion}

\subsection{Summary of Contributions}

This research presents a production-quality implementation of the local volatility framework with the following key achievements:

\begin{enumerate}
    \item \textbf{Arbitrage-Free Surfaces}: Zero post-enforcement violations across all tested equities
    \item \textbf{Excellent Calibration}: SVI RMSE consistently below 0.5\%
    \item \textbf{Numerical Stability}: No NaN/Inf values, robust derivatives
    \item \textbf{Systematic Optimization}: Documented improvement from 18\% to 0\% violations
    \item \textbf{Academic Rigor}: Literature-backed methodology suitable for publication
\end{enumerate}

\subsection{Academic Implications}

The explicit enforcement mechanism (Algorithm \ref{alg:enforcement}) addresses a gap in existing literature. While Andersen \& Brotherton-Ratcliffe (2005) and others acknowledge the need for enforcement in discrete implementations, few published works provide detailed algorithms. Our contribution makes the process transparent and reproducible.

\subsection{Practical Applications}

This framework is suitable for:
\begin{itemize}
    \item \textbf{Exotic Options Pricing}: Risk-neutral path simulation under local vol
    \item \textbf{Model Risk Assessment}: Comparison with stochastic vol models (Heston, SABR)
    \item \textbf{Hedging Strategies}: Delta/vega computation using local vol Greeks
    \item \textbf{Academic Research}: Time-series analysis of local vol dynamics
\end{itemize}

\subsection{Future Research Directions}

Several extensions merit investigation:

\begin{enumerate}
    \item \textbf{Extended SVI (eSSVI)}: Gatheral \& Jacquier (2014) provide an extended parametrization with additional flexibility for extreme strikes.
    
    \item \textbf{Stochastic Local Volatility}: Combine local vol with stochastic volatility to capture both smile and path-dependency.
    
    \item \textbf{Machine Learning Approaches}: Neural networks for arbitrage-free interpolation (Hernandez 2017).
    
    \item \textbf{High-Frequency Dynamics}: Intraday local vol surface evolution and microstructure effects.
    
    \item \textbf{Multi-Asset Extensions}: Local correlation surfaces for basket options.
\end{enumerate}

\subsection{Code Availability}

The complete implementation, including:
\begin{itemize}
    \item Python source code (\texttt{svi\_calibration.py}, \texttt{local\_vol\_builder.py})
    \item Jupyter notebook with detailed explanations (\texttt{local\_vol\_notebook.ipynb})
    \item Validation scripts and test cases
    \item Documentation (technical report, optimization summary)
\end{itemize}

is available at: \url{https://github.com/fdegirolamo/local-vol-surface}


\begin{thebibliography}{99}

\bibitem{BlackScholes1973}
Black, F., \& Scholes, M. (1973).
The pricing of options and corporate liabilities.
\textit{Journal of Political Economy}, 81(3), 637--654.

\bibitem{Dupire1994}
Dupire, B. (1994).
Pricing with a smile.
\textit{Risk Magazine}, 7(1), 18--20.

\bibitem{DermanKani1994}
Derman, E., \& Kani, I. (1994).
Riding on a smile.
\textit{Risk Magazine}, 7(2), 32--39.

\bibitem{Gatheral2004}
Gatheral, J. (2004).
A parsimonious arbitrage-free implied volatility parameterization with application to the valuation of volatility derivatives.
\textit{Presentation at Global Derivatives \& Risk Management, Madrid}.

\bibitem{Gatheral2006}
Gatheral, J. (2006).
\textit{The Volatility Surface: A Practitioner's Guide}.
John Wiley \& Sons.

\bibitem{GatheralJacquier2014}
Gatheral, J., \& Jacquier, A. (2014).
Arbitrage-free SVI volatility surfaces.
\textit{Quantitative Finance}, 14(1), 59--71.

\bibitem{AndersenBR2005}
Andersen, L., \& Brotherton-Ratcliffe, R. (2005).
Extended LIBOR market models with stochastic volatility.
\textit{Journal of Computational Finance}, 9(1), 1--40.

\bibitem{Rubinstein1994}
Rubinstein, M. (1994).
Implied binomial trees.
\textit{Journal of Finance}, 49(3), 771--818.

\bibitem{Savitzky1964}
Savitzky, A., \& Golay, M. J. E. (1964).
Smoothing and differentiation of data by simplified least squares procedures.
\textit{Analytical Chemistry}, 36(8), 1627--1639.

\bibitem{DavisHobson2007}
Davis, M. H. A., \& Hobson, D. G. (2007).
The range of traded option prices.
\textit{Mathematical Finance}, 17(1), 1--14.

\bibitem{Fengler2009}
Fengler, M. R. (2009).
Arbitrage-free smoothing of the implied volatility surface.
\textit{Quantitative Finance}, 9(4), 417--428.

\end{thebibliography}

\end{document}
